\documentclass[hidelinks]{article}
\usepackage[letterpaper,margin=1.0in]{geometry}
\usepackage[utf8]{inputenc}
\pagenumbering{arabic}
\usepackage{authblk}
\usepackage{graphicx}
\usepackage[singlelinecheck=false]{caption} % singlelinecheck makes single line caption left aligned instead of centered
\usepackage{subcaption}
\usepackage{amsmath}
\usepackage{fancyhdr}
\usepackage{longtable}
\usepackage{booktabs}
% hyperlinks
\usepackage{hyperref}
\usepackage{wrapfig}
\usepackage{xspace}
\usepackage{mathrsfs}
\usepackage{graphicx}
\usepackage{lipsum}
\usepackage{makecell}
 


\pagestyle{fancy}
\fancyhead[R]{\textbf{\stdpopsim with selection}}

% for highlighting text
\usepackage{xcolor}
\usepackage{soul}

% bibliography
\usepackage[
    backend=biber,
    natbib=true,
    style=authoryear,
    ]{biblatex}
\addbibresource{references.bib}

%\usepackage[round]{natbib}   % omit 'round' option if you prefer square brackets
%\bibliographystyle{plainnat}



\newcommand{\Stdpopsim}{\texttt{Stdpopsim}\xspace}
\newcommand{\stdpopsim}{\texttt{stdpopsim}\xspace}

%commands to format figure and table references in the supplement
\newcommand{\beginsupplement}{%
        \fancyhead[L]{Supplemental Material}
        \setcounter{table}{0}
        \renewcommand{\thetable}{S\arabic{table}}%
        \setcounter{figure}{0}
        \renewcommand{\thefigure}{S\arabic{figure}}%
     }
\newcommand{\stopsupplement}{%
        \setcounter{table}{0}
        \renewcommand{\thetable}{\arabic{table}}%
        \setcounter{figure}{0}
        \renewcommand{\thefigure}{\arabic{figure}}%
     }

\makeatletter
\newcommand{\labelname}[1]{\def\@currentlabelname{#1}}
\makeatother

\newcommand{\tskit}{\texttt{tskit}\xspace}

% add commands for msmc2, stairway plot, gone, and smc++
\newcommand{\msmc}{\texttt{msmc2}\xspace}
\newcommand{\stairway}{\texttt{stairwayplot}\xspace}
\newcommand{\gone}{\texttt{GONE}\xspace}
\newcommand{\smcpp}{\texttt{SMC++}\xspace}

% add commands for the DFE inference methods
\newcommand{\dfe}{\texttt{dfe-alpha}\xspace}
\newcommand{\polydfe}{\texttt{polyDFE}\xspace}
\newcommand{\dadicli}{\texttt{dadi-cli}\xspace}
\newcommand{\grapes}{\texttt{GRAPES}\xspace}

% add commands for the sweep detection methods
\newcommand{\sweepfinder}{\texttt{sweepfinder2}\xspace}
\newcommand{\diploshic}{\texttt{diploshic}\xspace}

% Avoid pandoc bug when there are lists in the body.
\providecommand{\tightlist}{%
\setlength{\itemsep}{0pt}\setlength{\parskip}{0pt}}

\newcommand{\rngcomment}[1]{\textcolor{red}{RNG: #1}}

\newcommand{\kelcomment}[1]{\textcolor{blue}{KEL: #1}}

\newcommand{\drscomment}[1]{\textcolor{purple}{DRS: #1}}

\newcommand{\igcomment}[1]{\textcolor{orange}{IG: #1}}

\title{Accesible, realistic genome simulation with selection using \stdpopsim}

% \author[1,+]{M. Elise Lauterbur}
% \author[3,*]{Ariella L. Gladstein}
\author[4,*]{Graham Gower}
\author[5*]{Nathaniel S. Pope}
\author[5*]{Murillo F. Rodrigues}
\author[5*]{Silas Tittes}
\author[34*]{Linh N. Tran}
\author[6]{Georgia Tsambos}

\author[2]{Maria Izabel A. Cavassim}
% \author[5,7]{Jeff Adrion}
% \author[5]{Saurabh Belsare}
% \author[8]{Arjun Biddanda}
% \author[5]{Victoria Caudill}
% \author[9]{Jean Cury}
% \author[10]{Ignacio Echevarria}
% \author[11]{Benjamin C. Haller}
% \author[12,13]{Ahmed R. Hasan}
\author[14,15]{Xin Huang}
% \author[16]{Leonardo Nicola Martin Iasi}
% \author[17]{Ekaterina Noskova}
% \author[18]{Jana Obšteter}
% \author[19]{Vitor Antonio Corrêa Pavinato}
% \author[20,21]{Alice Pearson}
% \author[22,23]{David Peede}
% \author[24]{Manolo F. Perez}
\author[5]{Chris C. R. Smith}
% \author[25]{Jeffrey P. Spence}
% \author[5]{Anastasia Teterina}

\author[5]{Scott T. Small}
\author[5]{Clara T. Rehmann}
% \author[26]{Per Unneberg}
% \author[27]{Juan Manuel Vazquez}
% \author[28]{Ryan K. Waples}
% \author[29]{Anthony Wilder Wohns}
% \author[30]{Yan Wong}
% \author[31]{Franz Baumdicker}
% \author[32]{Reed A. Cartwright}
% \author[33]{Gregor Gorjanc}
\author[4]{Kirk E. Lohmueller}
\author[34]{Ryan N. Gutenkunst}
\author[30]{Jerome Kelleher}
\author[35]{Aaron P. Ragsdale}

\author[37]{Daniel R. Schrider}
\author[38]{Ilan Gronau}
\author[5,36]{Peter L. Ralph}
\author[5]{Andrew D. Kern}


 \affil[*]{\small{These authors contributed equally to the paper.}}
% \affil[+]{\small{Corresponding authors: lauterbur@gmail.com ; ilan.gronau@runi.ac.il.}}
% \affil[1]{\small{Department of Ecology and Evolutionary Biology, University of Arizona, Tucson AZ 85719, USA}}
% \affil[2]{\small{Department of Ecology and Evolutionary Biology, University of California, Los Angeles, Los Angeles CA, USA}}
% \affil[3]{\small{Embark Veterinary, Inc., Boston MA 02111, USA}}
 \affil[4]{\small{Section for Molecular Ecology and Evolution, Globe Institute, University of Copenhagen, Denmark}}
 \affil[5]{\small{Institute of Ecology and Evolution, University of Oregon, Eugene OR 97402, USA}}
% \affil[6]{\small{School of Mathematics and Statistics, University of Melbourne, Australia}}
% \affil[7]{\small{AncestryDNA, San Francisco CA 94107, USA}}
% \affil[8]{\small{54Gene, Inc., Washington DC 20005, USA}}
% \affil[9]{\small{Université Paris-Saclay, CNRS, INRIA, Laboratoire Interdisciplinaire des Sciences du Numérique, UMR 9015 Orsay, France}}
% \affil[10]{\small{School of Life Sciences, University of Glasgow, Glasgow, UK}}
% \affil[11]{\small{Department of Computational Biology, Cornell University, Ithaca NY, USA}}
% \affil[12]{\small{Department of Cell and Systems Biology, University of Toronto, Toronto ON, Canada}}
% \affil[13]{\small{Department of Biology, University of Toronto Mississauga, Mississauga ON, Canada}}
\affil[14]{\small{Department of Evolutionary Anthropology, University of Vienna, Vienna, Austria}}
\affil[15]{\small{Human Evolution and Archaeological Sciences (HEAS), University of Vienna, Vienna, Austria}}
% \affil[16]{\small{Department of Evolutionary Genetics, Max Planck Institute for Evolutionary Anthropology, Leipzig, Germany}}
% \affil[17]{\small{Computer Technologies Laboratory, ITMO University, St Petersburg, Russia}}
% \affil[18]{\small{Agricultural Institute of Slovenia, Department of Animal Science, Ljubljana, Slovenia}}
% \affil[19]{\small{Entomology Department, The Ohio State University, Wooster OH, USA}}
% \affil[20]{\small{Department of Genetics, University of Cambridge, Cambridge, UK}}
% \affil[21]{\small{Department of Zoology, University of Cambridge, Cambridge, UK}}
% \affil[22]{\small{Department of Ecology, Evolution, and Organismal Biology, Brown University, Providence RI, USA}}
% \affil[23]{\small{Center for Computational Molecular Biology, Brown University, Providence RI, USA}}
% \affil[24]{\small{Department of Genetics and Evolution, Federal University of Sao Carlos, Sao Carlos 13565905, Brazil}}
% \affil[25]{\small{Department of Genetics, Stanford University School of Medicine, Stanford CA 94305, USA}}
% \affil[26]{\small{Department of Cell and Molecular Biology, National Bioinformatics Infrastructure Sweden, Science for Life Laboratory, Uppsala University, Husargatan 3, SE-752 37 Uppsala, Sweden}}
% \affil[27]{\small{Department of Integrative Biology, University of California, Berkeley, Berkeley CA, USA}}
% \affil[28]{\small{Department of Biostatistics, University of Washington, Seattle WA, USA}}
% \affil[29]{\small{Broad Institute of MIT and Harvard, Cambridge MA 02142, USA}}
\affil[30]{\small{Big Data Institute, Li Ka Shing Centre for Health Information and Discovery, University of Oxford, Oxford OX3 7LF, UK}}
% \affil[31]{\small{Cluster of Excellence - Controlling Microbes to Fight Infections, Eberhard Karls Universität Tübingen, Tübingen, Baden-Württemberg, Germany}}
% \affil[32]{\small{School of Life Sciences and The Biodesign Institute, Arizona State University, Tempe AZ, USA}}
% \affil[33]{\small{The Roslin Institute and Royal (Dick) School of Veterinary Studies, University of Edinburgh, Edinburgh EH25 9RG, UK}}
\affil[34]{\small{Department of Molecular and Cellular Biology, University of Arizona, Tucson AZ 85721, USA}}
\affil[35]{\small{Department of Integrative Biology, University of Wisconsin-Madison, Madison WI, USA}}
\affil[36]{\small{Department of Mathematics, University of Oregon, Eugene OR 97402, USA}}
\affil[37]{\small{Department of Genetics, University of North Carolina at Chapel Hill, Chapel Hill NC 27599, USA}}
\affil[38]{\small{Efi Arazi School of Computer Science, Reichman University, Herzliya, Israel}}

\date{\small{\today{}}}

\begin{document}

\maketitle


\section*{Abstract}
    \label{abstract}
    Natural selection is a fundamental evolutionary force that shapes patterns of genetic variation across species. 
    However, simulating realistic models that incorporate both selection and complex demographic histories
    is challenging, limiting our ability to benchmark statistical methods aimed at detecting selection and explore theoretical predictions.
    \stdpopsim is a tool for facilitating the simulation of a contiually expanding catalog of population genetic models.
    Here we present a major extension to the \stdpopsim{} framework that enables simulation of various modes
    of natural selection, including background selection, selective sweeps, and arbitrary distributions of fitness effects (DFE).
    This extension maintains \stdpopsim's core principles of reproducibility and standardization while adding support
    for species-specific genomic annotations and published DFE estimates. 
    We demonstrate the utility of this framework by benchmarking methods for demographic inference,
    DFE estimation, and selective sweep detection across different species and scenarios. 
    Our results demonstrate the robustness of demographic inference methods to selection on linked sites, 
    reveal the sensitivity of DFE inference methods to model assumptions, 
    and show how genomic features like recombination rate and the density of functional sequence influence power to detect selective sweeps.
    This extension to \stdpopsim{} provides a powerful new resource for the population genetics community
    to explore the interplay between selection and other evolutionary forces in a reproducible framework.

\section*{Introduction}
    \label{introduction}
    % natural selection
    Natural selection is a fundamental force in evolution, shaping the
    genetic diversity of populations and driving the adaptation of
    species to their environments. The effects of natural selection
    on genetic variation are complex, and can be difficult to disentangle
    from other evolutionary processes such as mutation, recombination,
    and genetic drift \citep[e.g.,][]{gillespie1991causes}.
    For instance, changes in population size can lead to fluctuations
    in genetic diversity across a recombining chromosome 
    that can mimic the effects of selection \citep{simonsen1995properties},
    and lead to spurious inferences about the strength and targets of genetic adaptation
    \citep{simonsen1995properties,akey2004population,nielsen2005genomic}.
    In turn, selection can confound our ability to infer demographic 
    history from allele frequencies \citep{ewing2016consequences,schrider2016effects} and
    estimates of inverse coalescent rate \citep{schrider2016effects, johri2021impact, cousins2024accurate}.
    Thus, it is imperative to jointly account for the effects of selection
    and demography when inferring evolutionary history from genetic data \citep{sheehan2016deep,johri2020toward}.
    However, this is a challenging task from a modeling perspective \citep{johri2022prospect}. % MFR this is a bit unclear

    % simulation
    To meet the growing need for the interpretation,
    analysis, and exploration of realistic and complex evolutionary models
    the field of population genetics has increasingly turned to simulation.
    Simulation in population genetics has a long history 
    including both backward-in-time coalescent simulations
    \citep{kingman1982genealogy,hudson1983testing, hudson1990gene}
    and forward-in-time simulations of complex demography and selection
    \citep[e.g.,][]{gillespie1984molecular,thornton2014c++, haller2019slim}.  \kelcomment{maybe also cite https://doi.org/10.1093/bioinformatics/btn522}
    The development of simulation tools has been driven by the need to
    understand the effects of complex evolutionary processes on genetic
    variation \citep[e.g.,][]{galloway2020few},\kelcomment{also cite: PMID: 32487519 } \drscomment{and maybe a few others? PMIDs: 24102784 and thornton2019polygenic} to provide a null model for hypothesis testing
    \citep[e.g.,][]{hudson1992statistical,hudson1994evidence,sabeti2002detecting},
    to explore the power and limitations of statistical methods \citep[e.g.,][]{przeworski2002signature},
    and increasingly to provide a basis for machine learning and other
    simulation-based inference methods \citep[e.g.,][]{beaumont2002approximate,pavlidis2010searching,lin2011distinguishing,kern2018diplos,mughal2019localizing,sanchez2021deep,wang2021automatic}. \kelcomment{also cite: PMID: 36617238 }
    However, joint simulation of complex demography and selection
    is challenging, requiring a deep understanding of the underlying
    evolutionary processes, and necessitating a number of parameter choices including
    the strength of selection, the
    recombination rate, and a parameterized demographic model.
    This is a daunting task for many researchers, and can be a barrier to
    the adoption of simulation-based methods in population genetics.
    Furthermore, different simulation and inference tools may use different structures
    and parameter scalings for their models,
    making it difficult to directly compare across these tools and the studies that use them.
    
    % stdpopsim
    In light of the complexities associated with estimating and simulating
    population genetic models, it is no surprise that
    a lingering challenge with simulation in population genetics has been
    reproducibility and the ability to share and compare results among 
    researchers \citep[e.g.,][]{ragsdale2020lessons}.
    This challenge has been addressed in part by the development
    of community resources for sharing and distributing simulation software
    via the \stdpopsim project \citep{adrion2020community}. \stdpopsim
    provides a standardized interface for accessing a wide range of
    population genetic models, and has begun to be be widely adopted by the community. %maybe add citations here?
    Yet, the original version of \stdpopsim did not include
    models of selection, which is a major limitation for more empirically motivated
    applications of population genetic simulation. In particular, modeling selection
    through simulation is critical for understanding processes such
    as adaptation \citep[e.g.][]{thornton2019polygenic,hartfield2024polygenic}, the effects of selective sweeps \citep[e.g.][]{braverman1995hitchhiking,fay2000hitchhiking,przeworski2002signature,przeworski2005signature,schrider2015soft}, and the impact of
    background selection on genetic diversity
    \citep[e.g.][]{charlesworth1993effect,charlesworth1995pattern,williamson2002genealogy,ewing2016consequences,torres2020temporal}.
    Ideally, one would like
    to have a single, unified framework for simulating both neutral and
    non-neutral evolutionary processes and to be able to compare the
    results of these simulations to empirical data in a manner that is
    both accessible to a wide range of researchers and highly reproducible. 
    Furthermore, the framework should include the complex realities of 
    genomes, including heterogenous recombination rates, 
    variation in the size and density of functional elements, and
    non-equilibrium demographic histories. Ideally, one should be able
    to model all of these features simultaneously using estimates and annotations obtained
    from a species of interest \citep[e.g.][]{schrider2020background,rodrigues2024shared}.

    % goals
    Here, we provide an overview of a major new addition
    to \stdpopsim---the inclusion of models of selection.
    We begin by describing the models of selection that we have implemented,
    as well as the parameter choices that are available to the user.
    This includes models of background selection, selective sweeps, and
    models of the distribution of fitness effects (DFE).
    Further we describe how these models can be combined with genomic
    annotations that are specific to a given species available
    through the \stdpopsim API, and combined with a parameterized model of
    demography and recombination to provide a realistic simulation of
    genetic variation in a population.
    We then provide a series of examples of how these models can be used
    to benchmark and compare the performance of different methods for
    using genetic data to infer demographic history, the distribution of fitness effects,
    and the targets of recent positive selection. 

\section*{Implementing Selection in \stdpopsim}
    \label{selection}

    % implementing selection
    \stdpopsim previously provided a wide range of models of neutral
    demographic history from roughly two dozen species \citep{lauterbur2023expanding}.
    This was accomplished through the use of a standardized interface
    and data structure that allows for the curated addition of new
    species and models, and can use either
    msprime \citep{Baumdicker2022} or SLiM \citep{haller2019slim}
    as the backend engine for efficient population genetic simulation.
    %IG: edited here to add details on the spec of the DFE and Annotation classes
    %moved below the information on how they are used in simulation. Please check!
    To extend \stdpopsim's functionality to include models of selection, 
    we introduce two new classes
    %an important new class representing a specfic distribution of fitness effects
    to the \stdpopsim API: the \texttt{DFE} class and the \texttt{Annotation} class.
    The \texttt{DFE} class provides an interface for specifying
    a distribution of the fitness effects (DFE) of new mutations.
    Each \texttt{DFE} object specifies a list of mutation types and
    their proportions, where 
    each mutation type is associated with a dominance coefficient
    (currently limited to values between 0 and 1, inclusive),
    the distribution of selection coefficients (e.g., fixed-value, normally-distributed, gamma-distributed),
    and the parameters of this distribution.
    Each \texttt{Annotation} object is typically associated with a collection of functional
    genomic elements whose coordinates are specified using a GFF3 annotation file. %
    %    
    For example, say that we wish to implement a model of selection consistent with the
    DFE inferred by Kim et al. (\citeyear{kim2017inference}) for exons in the human genome.
    The associated \texttt{DFE} object is defined with two mutation types,
    neutral and deleterious, corresponding to synonymous and non-synonimous sites.
    with their proportions set to their expected
    prevelanece in exons. 
    The neutral mutation type is associted with a fixed selection coefficient (0),
    and the deleterious mutation type is associated with a dominance coefficient of $0.5$ and
    a gamma-distributed selection coefficient
    with mean and shape as inferred by Kim et al. (\citeyear{kim2017inference}).
    \texttt{DFE} objects can also model a relationship between the selection and dominance coefficients \citep{huber2018gene}
    by defining ranges of selection coefficients and associating each range with a different dominance coefficient (see below).
    

    \begin{figure}[t!]
        \includegraphics[width=\linewidth]{figures/schematics/catalog.pdf}
        \caption{\label{fig:schematic}
        \textbf{(A)} A schematic of the \stdpopsim catalog and the different components that
        can be used in simulations. 
        \textbf{(B)} The python API and \textbf{(C)} the command line interface.
        The user specifies a species, a portion of
        the genome to simulate, and optionally a genetic map, a model of
        demography, and a model of selection that itself is composed of a
        distribution of fitness effects (DFE) and a set of functional
        annotations. }
    \end{figure}
    \kelcomment{For Figure 1 at the bottom, it says "Kyriazis et al. 2013". This should be "2023" instead}
    As with other aspects of \stdpopsim, we intend the selection models to be as biologically accurate as possible.
    Thus, the \stdpopsim catalog provides \texttt{Annotation} objects based on specices' publically available functional genomic elements 
    and \texttt{DFE} objects based on published DFE estimates (Figure \ref{fig:schematic}A).
    By using annotation objects, one can, for example, simulate populations where selected mutations (whose selection
    coefficients are drawn from a specified DFE) occur within annotation regions (e.g. exons), while all mutations
    occurring outside of these regions do not affect fitness.
    There are relatively few species with published DFE estimates in the literature.
    \stdpopsim currently has implemented DFEs for four species in the catalogue:
    \textit{Arabidopsis thaleana} \citep{huber2018gene}, \textit{Drosophila melanogaster} \citep{ragsdale2016triallelic,huber2017determining},
    humans \citep{huber2017determining,kim2017inference}, and the vaquita porpoise \textit{Phocoena sinus} \citep{robinson2022critically}.
    While the DFEs are from a limited number of sepecies, these DFEs can be applied to other species
    in the catalog if and when it is suitable. 
    For instance, one could simulate a butterfly species with a gamma-distributed DFE originally
    estimated from humans, or a human population with a DFE estimated from \textit{Drosophila}.
    Furthermore, the user can specify a custom DFE and provide their own annotations
    of functional elements to simulate selection in a species for which we do not yet have 
    a published DFE included in the catalog. This flexibility allows for a wide range of
    models of selection to be simulated. %, and for the results of these simulations to be
    %compared to empirical data or used as a null model for hypothesis testing.
    % DRS: commenting that stuff out because there are obviously other use cases that could be
    % listed here (e.g. ML/ABC, and if we don't know enough about a species to do that then
    % we probably have no business doing hypothesis testing either, which could be just as
    % misleading or maybe even more so), so the simplest solution is to cut this as it is
    % not needed here.
   
  
    From a user's perspective, a model of selection is specified by pairing a \texttt{DFE} object
    with a collection of genomic segments defined using an \texttt{Annotation} object.
    Multiple such pairings can be defined in one simulation
    to provide a rich model of how selection may vary along a chromosome.
    For example, coding sequences may be associated with one DFE and non-coding sequences in exons with another DFE.
    Simulations with selection are then implemented by specifying a species, 
    a portion of the genome to simulate,
    a genetic map if available, a model of demography, and pairings of DFEs with genomic segments
    using either the python API or the command-line interface (Figure \ref{fig:schematic}B-C).
    %IG: commented this out because I think it's covered by the text I wrote above
    %A \texttt{DFE} object can apply to specified intervals within a chromosome,
    %and can be combined with other \texttt{DFE} objects 
    Simulations with selection are generated
    using the SLiM simulation engine, which is
    %is perhaps
    the most flexible simulation engine for modeling selection available to date
    (msprime's selection models are relatively limited and thus not currently used by \stdpopsim).
    %IG: this is a nice statement, but I couldn't figure where to plug it after my changes so I'm leaving it commented
    %all in an automated and user-friendly fashion. 

    % sweep interface
    In addition to the \texttt{DFE} class, we have also implemented a class (\texttt{stdpopsim.ext.selective\_sweep})
    that enables selective sweeps to be simulated in \stdpopsim.
    This class augments a demographic model with an ``extended event''
    which conditions on the introduction of a selected mutation at a given time and position
    and with a given selection coefficient. Further, the user can specify the minimum frequency
    of the selected allele at the time of sampling. As these extended events are implemented
    on top of the existing \stdpopsim API, they can be combined with other models of selection
    and demography to provide a rich model of the combined effects of multiple disparate evolutionary processes
    on genetic variation in a population.
    

    % current numbers of DFEs / species / etc
    % DRS: are we giving this a version number?
    At this release the \stdpopsim catalog comprises 24 species (i.e. genome representations)
    for which we have included 28 demographic models, 37 genetic maps, and 7 DFEs. %someone should double check the numbers
    Additions to the catalog are ongoing---we point the reader to our previously published 
    report detailing this effort \citep{lauterbur2023expanding}. We welcome contributions from the
    community and encourage those interested in contributing to visit https://popsim-consortium.github.io/stdpopsim-docs
    or contact one of the PopSim Consortium members for guidance.
    \igcomment{Mention releas number above, update the num of species and models and maybe also mention the number of annotation objects.}


%IG: commented out this intro paragraph. Better to dive right in.
%\section*{Example Applications with Selection}
%    \label{applications}
%    In this section, we present a series of examples of how the new models of selection
%    in \stdpopsim can be used to benchmark different
%    methods for population genetic inference.
%    We focus on three main areas: demographic inference in the 
%    single and multi-population settings, inference of the DFE,
%    and the detection of selective sweeps, particularly in the context of realistic recombination maps and background selection.

    \section*{Inference of $N_e(t)$ in the context of selection}
    One of the most common applications of population genetic inference is to estimate
    the effective population size over time, $N_e(t)$, from genetic data. This can be done
    using a variety of methods, including the sequential Markovian coalescent
    \citep{li2011inference,Schiffels2020,terhorst2017robust}, 
    
    
    %\kelcomment{mabye also cite Gutenkunst 2009 Dadi paper here} 
    %\igcomment{I'm not sure it's all that relevant, since it doesn't strictly infer N(t)}
    %\drscomment{One could use dadi for that, but then we should also cite things like fastsimcoal and ABC, while explaining
    %that users can use those methods to infer piecewise-constant N(t) histories as well with a prespecified number of epochs
    %and priors around their params, but maybe that is too much of a digression here? I'm fine either way}
    through use of the site frequency spectrum (SFS) \citep{liu2020stairway},
    as well as through identity by descent information \citep{santiago2020recent}.
    Since all of these methods assume neutral evolution, they are typically applied to genomic
    segments that are not expected to be directly affected by selection (e.g., by masking exons
    and conserved elements).
    However, natural selection acting on linked sites---even those that are masked out of the
    analysis---can bias estimates of $N_e(t)$
    away from true census population sizes by increasing the 
    rate of coalescence in the genome \citep[e.g.][]{schrider2016effects}. 
    Using \stdpopsim, we can easily examine how different methods for inferring $N_e(t)$
    are influenced by selection on linked sites,
    which in the case of negative selection is commonly referred to as \emph{background selection}
    \citep{charlesworth1993effect,hudson1995deleterious}.

    
    \begin{figure}[b!]
        \centering
        \includegraphics[width=\textwidth]{figures/HomSap/OOA/estimated_Ne_t_final}
        \caption{
        \label{fig:human-demography}
        %citations aren't always allowed in captions. We should check with MBE.
        Inference of $N_e(t)$ from human genomes simulated under an out-of-Africa demographic model \citep{ragsdale2019models}
        with and without purifying selection on exons. 
        Rows correspond to the three extant populations in the simulation: CHB, CEU, and YRI.
        Columns correspond to the four inference methods:    
        \msmc \citep{Schiffels2020}, \stairway \citep{liu2020stairway}, \gone \citep{santiago2020recent}, and \smcpp \citep{terhorst2017robust}.
        Each plot depicts the inferred $N_e(t)$ on the three datasets simulated without selection (blue)
        and the three datasets simulated with the background effects of purifying selection on exons (orange),
        alongside the true values of $N_e$ used in simulation (black).
        }
    \end{figure}

    For this purpose, we simulated human genomes with and without natural selection.
    In both scenarios, we ran three replicate simulations
    under the out of Africa (OOA) demographic model of
    \citep{ragsdale2019models} using a genetic map from the HapMap Project \citep{international2007second} (see \textbf{Methods}).
    Simulations with selection were implemented by modeling purifying selection on nonsynonymous mutations
    % IG: moved details to methods
    %, defined from the HAVANA group release 104 for the human genome,
    using a DFE inferred by \cite{kim2017inference}.
    Following the simulation, inference of $N_e(t)$ was conducted from each of the six simulated datasets using four methods: 1) \msmc \citep{Schiffels2020}, 
    2) \stairway \citep{liu2020stairway}, 3) \gone \citep{santiago2020recent}, and 4) \smcpp \citep{terhorst2017robust}.
    % IG: I commented this out because it appeared distracting. We can add a note in the end of this analysis
    % However, as a thorough benchmarking is not the main purpose of this manuscript, we saved computational resources by not conducting more simulation replicates.

    Although the estimates produced by the four methods differ from one another, 
    for each method the estimates produced from data with selection (Figure
    \ref{fig:human-demography}; orange) and without selection (Figure
    \ref{fig:human-demography}; blue) are fairly similar.
    Thus, this particular parameterization of background selection does not
    appear to notably bias the estimates produced by these methods.
    As noted in previous studies that compared demography inference methods
    on neutral simulations \citep{adrion2020community},
    we see that each method has its strengths and weaknesses.
    Methods based on the sequential Markovian coalescent (\smcpp and \msmc)
    appear to produce the most accurate trajectories of $N_e(t)$ overall
    in this setting, although they tend to under-estimate the population
    bottleneck associated with the population split.
    On the other hand, \stairway, which utilizes the site frequency spectrum,
    produces more noisy estimates,
    but it does not appear to under-estimate the population bottleneck.
    The inference of \gone, which utilizes identity by descent,
    is targeted toward demographic changes in more recent time frames ($\sim 200$ generations),
    explaining its noisy estimates for all but the most recent time periods.
    
    


%IG: commented old version of figure here
%    \begin{figure}[t]
%        \centering
%        \includegraphics[width=\textwidth]{figures/HomSap/OOA/estimated_Ne_t_final}
%        \caption{
%        \label{fig:1pop-human-demography}
%        Inference of $N_e(t)$ from a human genomes simulated under an out-of-Africa demographic model \citep{ragsdale2019models}
%        with and without purifying selection on exons. The left column shows estimates of $N_e(t)$ obtained from genomes simulated
%        without selection, while the right column shows estimates of $N_e(t)$ from genomes simulated with a gamma-distributed   
%        DFE acting on exons. In each panel we show the true $N_e(t)$ in black, and the estimated $N_e(t)$ from four methods:    
%        \msmc \citep{Schiffels2020}, \stairway \citep{liu2020stairway}, \gone \citep{santiago2020recent}, and \smcpp \citep{terhorst2017robust}.
%        }
%    \end{figure}

    To expand these observations to additional species and to highlight the ease of comparison enabled by \stdpopsim,
    we performed a similar benchmark using the vaquita porpoise \textit{Phocoena sinus}.
    We simulated 100 genomes under a two-epoch demographic model inferred for vaquita in \textcite{robinson2022critically},
    and a constant recombination rate across the genome (see \textbf{Methods}).
    In simulations with selection, we applied the DFE inferred by \textcite{robinson2022critically} to exons
    annotated from the vaquita genome assembly.
    This DFE model implements a relationship between selection and dominance 
    in which very deleterious mutations (with selection coefficient $s<-0.1$) are fully recessive
    (dominance coefficient $h=0$) and weakly deleterious mutations ($s\geq -0.001$) are nearly
    additive ($h=0.4$; see \textbf{Methods}).
    Here, we see that the site-frequency-based method (\stairway) produces more accurate
    inference than the two methods based on the sequential Markovian coalescent (\smcpp and \msmc; see Figure \ref{fig:vaquita-demography}).
    Reagrdless of these differences, we see that background selection again does not
    appear to considerably influence the inference of any of the four methods for this species,
    which has a very different evolutionary history than humans.
    
    
    % MFR: Should we say somewhere that people use to expect from BGS? That is, that it can be seen as a reduction in local rates of coalescences and thus that we would expect these methods to be somewhat downwardly biased in the presence of BGS?
    % IG: we can address this in the discussion, or in the revised text above.


%\begin{figure}[t]
%    \centering
%    \includegraphics[width=\textwidth]{figures/PhoSin/Vaquita2Epoch_1R22/estimated_Ne_t_final}
%    \caption{
%    \label{fig:1pop-vaquita-demography}
%    Performance of methods to infer $N_e(t)$ from simulations of the vaquita porpoise genome under a single population
%    model of declining population size \citep{robinson2022critically} with and without background selection on exons. 
%    The left panel shows estimates of $N_e(t)$ from simulations
%    without selection, while the right panel shows estimates of $N_e(t)$ from simulations with a gamma-distributed   
%    DFE acting on exons. In each panel we show the true $N_e(t)$ in black, and the estimated $N_e(t)$ from four methods:    
%    \msmc \citep{Schiffels2020}, \stairway \citep{liu2020stairway}, \gone \citep{santiago2020recent}, and \smcpp \citep{terhorst2017robust}.
%
%    }
%\end{figure}

\section*{Estimation of the Distribution of Fitness effects}
    \label{dfe}
    Another common application of population genetic inference is to estimate the distribution of fitness effects (DFE) of new mutations
    from genetic data. The new framework of modeling selection in \stdpopsim makes it an ideal tool for easily benchmarking and comparing
    the performance of different methods for inferring the DFE. 
    %IG: I don't think we need this intro.
    %We performed simulations using the 
    %human and vaquita porpoise models described above, with complex demographic histories and models of selection acting on exons.
    %
    A challenge when comparing different inferred and simulated distributions of the selection coefficient ($s$) is differing convention regarding its definition.
    The simulator SLiM and the inference tool \grapes define the selection coefficient $s$ such that a homozygote has fitness $1+s$.
    On the other hand, \dadicli and \polydfe define the selection coefficient such that a homozygote has fitness $1+ 2s$.
     %https://academic.oup.com/genetics/article/207/3/1103/5930676
    Moreover, inferred distributions of selection coefficients are typically scaled by an inferred ancestral
    effective population size $N_e$, and different methods assume different scaling.
    \dadicli infers the distribution of $2 N_e s$ while \polydfe and \grapes infer the distribution of $4 N_e s$.
    %https://academic.oup.com/genetics/article/207/3/1103/5930676, https://academic.oup.com/genetics/article/216/2/559/6066183
    For comparison, here we have normalized all inference results to the convention of SLiM (see \textbf{Methods}).
    % Add citation to fastDFE as the successor to polyDFE: https://academic.oup.com/mbe/article/41/5/msae070/7641109 ?

    \begin{figure}[b!]
        \centering
        \includegraphics[width=\linewidth]{figures/HomSap/OOA/HomSap_discrete_DFE}
        \drscomment{I think it would be good to increase the font size for the A,B,C panel legends and also
        draw a horizontal line above panel C so it is clear that all three plots at the bottom are part of C.}
        \caption{Inferred versus simulated DFEs from simulated human genomes.
        Simulations were performed using a human out-of-Africa demographic model with a gamma-distributed DFE
        acting on exons (see \textbf{Methods}).
        DFE is inferred separately for each of the three extant populations (CEU, CHB, and YRI)
        by three different methods: \grapes, \polydfe , and \dadicli.
        (A-B) Mean absolute value of selection coefficient ($\lvert E(s) \rvert $) and shape parameter are
        shown for each DFE inferred from all three simulated datasets,
        with median values marked by horizontal bars
        and simulated values represented by dashed horizontal lines.
        (C) Binned distribution of $s$ implied by the average DFE inferred from the three simulated datasets (averaging the inferred gamma parameters);
        white bars represent the distribution used in the simulation.
        }
        \label{fig:homsap-dfe.ooa}
    \end{figure}

 

    We start by examining DFE inference from human genomes simulated under an out-of-Africa demographic model
    with selection acting on exons (see \textbf{Methods}).
    In this setting, the DFE is inferred separately for each of the three simulated populations
    using three different DFE inference methods: \dadicli \citep{Huang2023}, \polydfe \citep{tataru2020polydfe}, 
    and \grapes \citep{galtier2016adaptive}.
    In each case we note that the DFE is estimated from segregating sites only,
    without the use of substitution data, as \stdpopsim's models currently do not
    contain outgroup populations or species that can be used to estimate divergence.
    %IG: no need to map figures. Just describe results and cite figs
    %Figure \ref{fig:homsap-dfe.constant} shows estimates from a single, constant size population of humans, 
    %while Figure \ref{fig:homsap-dfe.ooa} shows estimates from a model of human out-of-Africa demography.
    %In the out-of-Africa model (Fig.~\ref{fig:homsap-dfe.ooa}) the performance of the methods is more mixed. 
    Overall, inferred DFEs varied across populations and inference methods (Figure \ref{fig:homsap-dfe.ooa}).
    Note that while the estimation accuracy for the DFE parameters themselves (Figure \ref{fig:homsap-dfe.ooa}A-B) is useful,
    examining the difference between the actual and inferred distributions of $s$ can be more helpful for showing how well a method
    has recovered the true DFE (Figure \ref{fig:homsap-dfe.ooa}C).
    The distributions of $s$ inferred by \polydfe and \dadicli had higher mean values (in absolute value)
    but smaller shape parameters, while \grapes inferred distributions with lower mean values and shape parameters.
    As a result, \polydfe and \dadicli over-estimate the fraction of mutations under strong selection ($s<-0.1$),
    while \grapes considerably over-estimates the fraction of mutations under weak selection ($s\geq-0.0001$).
    % IG: commented this out to simplify the basic observation
    % and under-estimates the fraction of mutations with $s<-0.01$.
    
    Population history appears to influence different methods in slightly different ways.
    For instance, \dadicli and \grapes produce more accurate estimates from the YRI genomes 
    (relative to the other two populations), whereas \polydfe
    produces its most accurate estimate from the CHB genomes.
    To factor out the influence of demographic changes, we also analyzed genomes generated using similar
    simulations but with no population size changes (Figure \ref{fig:homsap-dfe.constant}).
    In this simpler setting, \polydfe overlapped the true mean selection coefficient and shape parameter, whereas 
    the other methods do not. 
    \kelcomment{From looking at the SI Figure, these differences seem really slight. Also, with only 3 replicates, I think it's hard to conclusively say that some methods overlap the true values while others do not. I might instead just say here that all methods are close, but deviate slightly in different ways.
    \grapes exhibits similar biases without population size changes as in the simulations with more complex demography,
    but with milder under-estimation of the shape parameter, but \dadicli shows an inverse pattern
    with overestimated shape parameter and slightly underestimated mean. 
    
    \begin{figure}[b!]
        \centering
        \includegraphics[width=0.8\textwidth]{figures/PhoSin/Vaquita2Epoch_1R22/PhoSin_Vaquita2Epoch_1R22_Gamma_R22_Phocoena_sinus.mPhoSin1.pri.110_exons_DFE_plot.pdf}
        \caption{
        \label{fig:vaquita-dfe}
        Inferred versus simulated DFEs from simulated vaquita genomes.
        Simulations were performed using a two-epoch model of vaquita porpoise demography with a gamma-distributed DFE
        acting on nonsynonymous mutations with a relationship between the selection coefficient ($s$) and dominance coefficient ($h$) (see \textbf{Methods}).
        The DFE is inferred by analyzing all simulated genomes jointly by one of three different methods:
        \grapes, \polydfe, and \dadicli.
        (A-B) Mean absolute value of selection coefficient ($\lvert E(s) \rvert $) and shape parameter are
        shown for each DFE inferred from all three simulated datasets,
        with median values marked by horizontal bars
        and simulated values represented by dashed horizontal lines. 
        (C) Binned distribution of $2hs$ implied by the simulated DFE compared with
        the distribution of $s$ implied by the average DFE inferred for each method from the three simulated datasets (averaging the inferred gamma parameters).
        The distribution of $2 h s$ is multimodal because of the simulated relationship between $h$ and $s$ (see text).
        }
    \end{figure}
    
    %IG: describe experiment and cite figure later
    \kelcomment{Throughout this section, we talk about "mutations in exons". Is it more accurate to say "nonsynonymous mutations"? In other words, the simulations of exons include by synonymous and nonsynonymous mutations, but only the nonsynonymous ones experience selection and only nonsynonymous variants are used in DFE inference. Did I get that right? Maybe saying that clearly here would be good}
    %In Figures \ref{fig:vaquita-dfe.constant} and \ref{fig:vaquita-dfe} we show a comparison of estimates
    Next, we benchmarked these three DFE-inference methods on simulations of the vaquita porpoise genome.
    Recall that these simulations apply a gamma distribution of selection coefficients to nonsynonymous mutations 
    and a relationship between the selection and dominance coefficients (see \textbf{Methods}).
    In this setting, all methods perform uniformly worse than in the human genome simulations,
    with consistent underestimation of the mean selection coefficient    
    and overestimation of the shape parameter (Figure \ref{fig:vaquita-dfe}A-B). 
    Similar patterns are observed when re-running the simulations with a simpler demographic model
    without population size changes (Figure \ref{fig:vaquita-dfe.constant}).
    We suspect that this is due to the fact that all DFE inference methods used here assume an additive model ($h = \frac{1}{2}$),
    while mutations in the simulated genomes were more recessive ($h < \frac{1}{2}$), and their dominance decreased with the strength of selection. 
    To examine the influence of this model misspecification, we compared the inferred distributions of $s$ to the 
    simulated distribution of $2hs$, which would have been equal to $s$ under an additive model (Figure \ref{fig:vaquita-dfe}C).
    Reassuringly, all three methods infer distributions that appear to fit the main mode of the simulated distribution of $2hs$.
    This is likely because deleterious alleles are typically at low frequency and thus heterozygous, 
    where their selective effect is $h s$ rather than $s$.
    %KEL: I moved the last sentence from here to the Discussion and fleshed it out

\section*{Detection of Selective Sweeps}
    \label{sweeps}
    We next highlight how \stdpopsim can be used to benchmark and compare
    the performance of different methods for detecting selective sweeps from genetic data.
    Because simulation of selective sweeps is very time consuming and evaluation of statistical power requires
    many replicate simulations, we focused on human chromosome 1 and generated replicate simulations on 5cM segments
    of the chromosome in 100 different locations. % (see \textbf{Methods}).
    Selective sweeps were simulated by introducing a beneficial mutation in the center of the 5cM segment
    with a moderately strong selection coefficient ($s = 0.03$; $2Ns \sim 600$; see \textbf{Methods}).
    Each segment was simulated 1000 times, while keeping only replicates in which the selected allele reached a frequency of 0.95 or greater.
    These sweeps were modeled in the context of the three population out-of-Africa model from \cite{gutenkunst2009inferring}
    and background selection from deleterious mutations in exons.
    Sweep-detection methods typically base their inference on checking whether a test statistic
    exceeds a certain threshold, which is determined according to some empirical null distribution.
    In our simulations, we considered two different null distributions:
    one based on neutral simulations, and another based on simulations with background selection
    from deleterious mutations in exons (see \textbf{Methods}).
  
     %sweeep power figure
    \begin{figure}[t!]
        \centering
        \includegraphics[width=0.8 \textwidth]{figures/sweeps/chr1_power.pdf}
        \caption{
        Power to detect selective sweeps at 100 locations along human
        chromosome 1. Genomic segments were simulated with sweeps under a three population out-of-Africa model
        \citep{gutenkunst2009inferring} and with background selection from deleterious mutations in exons.
        %IG: seems like methods-level detail
        %Single beneficial mutations were introduced at each location with a selection
        % coefficient of $s = 0.03$, and only sampled if they reached a terminal
        % frequency of 0.95 or greater.
        Three methods for detecting sweeps were applied to simulated data:
        1) \sweepfinder \citep{degiorgio2016sweepfinder2} (top row),
        2) \diploshic \citep{kern2018diplos} (middle row),
        and 3) reduced diversity ($\pi$) (bottom row).
        Power (true positive rate) is shown for these methods for the CEU and YRI
        samples (left and right respectively).
        The thresholds of the test statistics were set to control for
        $5\%$ false positive rate under a neutral null model (blue)
        and a null model with background selection from deleterious mutations in exons (red).
        Also shown are heatmaps of exon density and local recombination rate along the
        chromosome (bottom).
        \igcomment{In fig, replace CLR with \sweepfinder; also in figs 6 S5 S6}
        }
        \label{fig:chr1_power}
    \end{figure}
   
    We evaluated the statistical power to detect selective sweeps for three methods:
    1) \sweepfinder \citep{degiorgio2016sweepfinder2},
    2) \diploshic \citep{kern2018diplos},
    and 3) reduced diversity ($\pi$).
    For each method, we determined a threshold for its test-statistic 
    using each of the two null models separately (neutral and BGS) and controlling 
    for a $5\%$ false positive rate.
    We then applied each method to data simulated with sweeps for two sampled populations (YRI and CEU)
    in each of the 100 locations along human chromosome 1 (Figure \ref{fig:chr1_power}).
    First, we notice that power to detect sweeps is slightly lower when the test statistic threshold
    is determined using simulated data with background selection (BGS null model).
    %This observation is consistent with what was previously shown by \cite{schrider2020background},
    %but the effect is relatively minor. DRS: cutting this because my paper showed that power decreases when our test set has BGS and our training set does not, so not quite the same.
    This is likely because the strength of selection acting on exons is relatively weak, and the
    sweep detection methods we have used are relatively robust to the effects of background selection. 
    Second, the power to detect sweeps is generally higher in the YRI sample
    than in the CEU sample, which is consistent with the fact that YRI has a larger ancestral effective population size and
    thus both stronger selection and more variation to detect a sweep \citep[e.g.,][]{simonsen1995properties}.
    Further, the CEU population experienced the out-of-Africa bottleneck
    that may mute the signal of a sweep.  
    Finally and most importantly, the power to detect sweeps varies considerably along the chromosome,
    and this variation is much greater than the variation in power between CEU and YRI or between
    null models. This is likely due to the effects of recombination rate heterogeniety (Figure \ref{fig:chr1_power}).% as well
    %as the density of functional elements along the chromosome.
    %DRS: cut the density of func elements because that stuff doesn't seem to matter much here

    We compared the overall performance of the three sweep-detection methods using
    receiver operating characteristic (ROC) curves (Figure \ref{fig:roc-curves}).
    Different methods perform better for each of the sampled populations,
    with \sweepfinder performing better than the other two on the CEU sample,
    and \diploshic performing better on the YRI sample.
    This is may result from the fact that we trained \diploshic on simulations under a constant size population,
    which more closely matches the demographic history of YRI, than that of CEU, which 
    is characterized by a strong recent bottleneck followed by population expansion.
    \drscomment{note: I am couching the language above a bit because we have previously shown that shic doesn't suffer in the sweep vs. nonsweep question when doing other train-on-constant-test-on-nonequilib benchmarks. Could be something else going on here (or some interaction between demog misspec and rec rate variation), but in any case I am not certain of the cause so I am toning down our certainty slightly.}
    Interestingly, the simple method that uses reduced diversity ($\pi$) as an indicator
    for a selective sweep performs reasonably well overall (especially on the CEU sample).
    However, its accuracy is much more variable along the chromosome compared to the other two method (see Figure \ref{fig:chr1_power}),
    and for quite a few locations along the chromosome it has power below $50\%$
    when applied to both sample sets.
    In that sense, \sweepfinder appears to have the least variation in power along the chromosome or across populations.




    \begin{figure}[b!]
        \centering
        \includegraphics[width=0.8 \textwidth]{figures/sweeps/relationship_power_cM.pdf}
        \caption{
        Power to detect selective sweeps 
        %IG: shortened the title line
        %at 100 locations along human
        %chromosome 1 under a three population out-of-Africa model
        %\citep{gutenkunst2009inferring}  when using null models with and
        %without purifying selection
        as a function of local recombination rate.
        This figure shows the same power estimates shown in Figure \ref{fig:chr1_power},
        but with the 100 genomic segments sorted according to their
        average recombination rates instead of position along chromosome 1.
        \igcomment{Specify how the curves were computed}
        %IG: copoed from the caption of fig 5
	Genomic segments were simulated with sweeps under a three population out-of-Africa model
        \citep{gutenkunst2009inferring} and with background selection from deleterious mutations in exons.
        Three methods for detecting sweeps were applied to simulated data:
        1) \sweepfinder \citep{degiorgio2016sweepfinder2} (top row),
        2) \diploshic \citep{kern2018diplos} (middle row),
        and 3) reduced diversity ($\pi$) (bottom row).
        Power (true positive rate) is shown for these methods for the CEU and YRI
        samples (left and right respectively).
        The thresholds of the test statistics were set to control for
        $5\%$ false positive rate under a neutral null model (blue)
        and a null model with background selection from deleterious mutations in exons (red).
        }
        \label{fig:power-recomb}
    \end{figure}

    To further dissect the reasons for high variance along the chromosome in power to detect sweeps,
    we examined the relationship between power and local recombination rate for our sweep simulations ( Figure \ref{fig:power-recomb}).
    We see that power to detect sweeps is
    a decreasing function of local recombination rate for \sweepfinder and reduced diversity,
    which is consistent with sweeps having a larger genomic footprint in regions of low
    recombination.
    \diploshic on the other hand shows a slight increase in power with recombination rates,
    probably because it is designed to distinguish regions where a sweep happened from regions linked to a sweep,
    and if the recombination rate is too low (or selection too strong) then the window examined by \diploshic may
    not be large enough to exhibit enough of a recovery in patterns of diversity flanking the sweep.
    These results confirm that reduced genetic diversity performs well as an indicator for selective sweeps only in regions with
    low recombination rate.
    The other two methods, which were specifically designed for sweep detection, are considerably more robust 
    to the effects of recombination rate than the simple indicator of reduced genetic diversity.
    These methods are particularly robust to recombination rate variation when applied to 
    the YRI sample, especially when using a neutral null model to calibrate the test statistics.
    The other null model, which assumes background selection from exons,
    appears to consistently lead to reduced power for detecting sweeps regardless of the recombination rate.
    Interestingly, we do not see a clear correlation between power to detect sweeps and
    local exon density (Figure \ref{fig:power-exon}).
    This is somewhat surprising given that background selection is expected to somewhat interfere with the sweep signature.
    This might suggest that the effects of realistic background selection on genome variation
    may not be as similar to those of sweeps as previously thought \citep{schrider2020background}.

\section*{Discussion}
    \label{Discussion}
    In this paper, we have presented an important new addition to the \stdpopsim{} library
    that allows for simulating genetic variation in the presence of selection.
    We have demonstrated the utility of this new API by showing how it can be used to benchmark
    the performance of different methods for demographic history inference, the inference of the distribution
    of fitness effects of new mutations, and the detection of selective sweeps.
    In general, this is an important step forward for the field of population genetics,
    as it allows for a wide range of models of selection to be easily simulated and compared to
    empirical data, with reproducibility, computational efficiency, and rigor.

    One important finding we made using the new stdpopsim selection API is that inference of the DFE is sensitive to the dominance of deleterious mutations.
    All three of the inference methods of the DFE applied here assume that deleterious mutations act in an additive fashion ($h=0.5$) by default.
    However, the simulated DFE for the vaquita genome included highly recessive mutations.
    Specifically, these simulations used a DFE where mutations with $s<-0.01$ are quite recessive, with $h<=0.01$ and where mutations with $-0.01<s<-0.001$ have $h=0.1$.
    We found that the three DFE inference methods underestimated $E[s]$ and overestimated the shape parameter (Fig 4).
    This performance is not surprising as decreasing $h$ for a given distribution of $s$ will lead to a decrease in $hs$,
    or an apparent decrease in the amount of selection.
    Fitting a mis-specified DFE that assumes $h=0.5$ will then lead to underestimates of $E[s]$.
    Indeed, the distribution of $hs$ is much more accurately estimated (Fig 4b).
    Overall, these results are consistent with prior work showing that it is challenging to separately infer the DFE of $s$ along with $h$ (cite PMID: 24830675; PMID: 39302992; PMID: 34951958 )
    and that a large proportion ($>5\%$) of recessive deleterious mutations can confound inferences of $s$ (PMID: 37074880).
    Given the plethora of evidence for recessive deleterious mutations (cite PMID: 4630587, PMID: 39114967 , PMID: 30013096 , PMID: 21098719 ),
    these results suggest the need for care in DFE inference.
    Kyriazis and Lohmueller (PMID: 39302992 ) inferred the DFE of $s$ for nonsynonymous mutations in humans when considering different values of $h$.
    Intriguingly, additive models show a satisfactory fit to the SFS of nonsynonymous variants in humans (Kyriazis et al) and more complex models of dominance do not substantially improve the fit.
    Nevertheless, models including recessive deleterious mutations, like those simulated here for the vaquita genome, show a similar fit to the SFS and may be more biologically realistic than fully additive models,
    given the evidence for the presence of recessive deleterious mutations.
    More broadly, new computational approaches are needed to precisely quantify the DFE and the covariance of $h$ and $s$.

    Regarding sweep detection, the inclusion of selection models into \stdpopsim allowed us to easily compare the
    effectiveness of different methods on data reflecting the demographic history, recombination rate variation,
    and genomic locations of negatively selected sites present in human populations. These results clearly demonstrate
    the impact of recombination rate variation, and underscore the importance of training methods on data with variable
    recombination rates (e.g. following Schrider and Kern \citeyear{schrider2017soft}). In addition, our results
    illustrate a strength of likelihood methods like \sweepfinder, which do not require a predefined window size and
    thus will perform better than \diploshic---which only examines relative summarie of diversity across a window instead
    of the actual values of these diversity measures---on very strong sweeps that effect diversity across a wide stretch of
    chromosomes that extends far beyond \diploshic's window size. \drscomment{Maybe this isn't the coolest stuff ever but
    I thought it might be worth saying something about our sweep detection stuff in the Discussion.}

    Although we have shown how the new \stdpopsim API can be used to benchmark the performance of different methods,
    the benchmarks shown here merely scratch the surface of what can be done with the new API.
    We envision that, moving forward, many researchers will use
    the new API to simulate data under a wide range of models of selection, demography, and genome architecture, and to develop
    new methods for the analysis of genetic data in the presence of selection. 
    The additions to \stdpopsim described here further expand this tool's utility for both 
    characterizing the patterns of diversity expected under a wide range of population genetic
    models, and for benchmarking methods for the analysis of genetic data
    on common ground
    %, using the same simulations and parameterizations of models of selection
    % DRS: this sentence got pretty bloated with my edits so I commented part of it out
    in a way that was not possible before.

\section*{Methods}
    \label{methods}

    \subsection*{Whole genome simulations}
    In this study we performed simulations of the human and vaquita porpoise genomes
    using the \stdpopsim{} framework.
    All simulations were executed using the SLiM engine with a burn-in of $2N$ generations
    (where $N$ is the initial population size), and a scaling factor of 2 (to reduce running time).
    Each simulation was implemented using a relatively short python script, similar to the one shown in Figure 1B
    (see \textbf{Data availablity} below).
    \igcomment{We should add a link to the analysis2 repo in data/code availablity section}
    %https://github.com/popsim-consortium/analysis2/tree/main
    % DRS: Maybe we should say something about how easy it was to do these simulations, and maybe even show
    % some example code somewhere to prove our point? Could help emphasize the utility of this resource.
    % IG: added a note about this above.
   
    \subsubsection*{Simulations of complete human genomes}
    Simulation of human genomes used a genetic map from the HapMap Project
    \citep{international2007second} (\stdpopsim label \texttt{HapMapII\_GRCh38}) and
    the out of Africa (OOA) demographic model with archaic admixture from
    \cite{ragsdale2019models} (\stdpopsim label \texttt{OutOfAfricaArchaicAdmixture\_5R19}).
    Simulations with selection additionally applied a DFE model with gamma-distributed selection coefficients inferred by
    \cite{kim2017inference} (\stdpopsim label \texttt{Gamma\_K17}) to exons annotated by the HAVANA group release 104
    (\cite{ensembl2018}; \stdpopsim label \texttt{ensembl\_havana\_104\_exons}).
    Each simulation generate 100 diploid genomes from each of the three extant human populations (YRI, CEU, and CHB),
    with each genome consisting of 22 autosomes.
    Using these settings, we generated three replicate datasets without selection
    and three replicate datasets with selection.
    Three additional datasets, used in Figure \ref{fig:homsap-dfe.constant},
    were simulated with the same model of selection and a simple demographic model with one population with a constant size of $10,000$.
    Each of these simulations generated 100 diploid genomes.
    
    \subsubsection*{Simulations of complete vaquita porpoise genomes}
    Simulation of vaquita porpoise genomes used 
    the two-epoch demographic model from \cite{robinson2022critically} (\stdpopsim label \texttt{Vaquita2Epoch\_1R22}).
    Since no genetic map is currently available for this species, we assumed a constant recombination rate of  $10^{-8}$
    across all chromosomes \citep{morin2021}.
    Simulations with selection additionally applied a DFE model with gamma-distributed selection coefficients inferred by
    \cite{robinson2022critically} (\stdpopsim label \texttt{Gamma\_R22}) to exons annotated from the vaquita genome assembly
    (\cite{morin2021}; \stdpopsim label \texttt{Phocoena\_sinus.mPhoSin1.pri.110\_exons}).
    This DFE model implements a relationship between selection and dominance as follows:
    very deleterious mutations (with $s<-0.1$) were set to be recessive ($h=0$),
    mutations with $s\in [-0.1,-0.01)$ were set to have a dominance coefficient of $h=0.01$,
    mutations with $s\in [-0.01,-0.001)$ were set to have a dominance coefficient of $h=0.1$,
    and very mild mutations (with $s\in [-0.001,0)$) were set to have a nearly additive effect ($h=0.4$).
    This approach is similar to what was used in the simulations cunducted by \textcite{robinson2022critically}
    and is easily implemented in \stdpopsim's \texttt{DFE} class.
    Each simulation generate 100 diploid genomes,
    with each genome consisting of 21 autosomes.
    Using these settings, we generated three replicate datasets without selection
    and three replicate datasets with selection.
    Three additional datasets, used in Figure \ref{fig:vaquita-dfe.constant},
    were simulated with the same model of selection and a simple demographic model with one population with a constant size of $3,500$.
    \igcomment{If we end up using Fig. S3, then we should add a line describing it's simulation here.}



    \subsection*{Inference of $N_e(t)$}
    %We assessed the accuracy of demographic inferences from
    %\msmc \citep{Schiffels2020}, \stairway \citep{liu2020stairway}, \gone \citep{santiago2020recent}, and \smcpp \citep{terhorst2017robust}.
    Inference of $N_e(t)$ was conducted on simulated datasets after masking non-recombining portion
    of the genome (to avoid bias caused by extreme recombination rate variation)
    and exons (to avoid the direct effect of natural selection).
    Thus, inference from datasets simulated with selection was not influenced by direct selection
    acting on exons, but it was influenced by selection acting on linked sites (background selection).    
    %IG: I commented because it doesn't seem relevant for methods
    %Comparisons between inferences made with and without selection were 
    %of particular interest.
    For \msmc \citep{Schiffels2020}, we used a random sample of $N=6$ simulated genomes with 20
    iterations of the EM algorithm.
    For \gone \citep{santiago2020recent}, we set \texttt{max\_snps} to 500,000,
    the number of generations to 2,000, and the number of bins to 400, all representing default settings.
    For \stairway \citep{liu2020stairway} and \smcpp \citep{terhorst2017robust} we used the default settings.
    \igcomment{Nothing more to say about \stairway plot and \smcpp?}
    

    \subsection*{DFE inference}
    \igcomment{I commented all the text that was here because it wasn't relevant to the methods. We might want to add it to the discussion.
    Kirk also noted that a lot of methods-level details are missing:
    how big was the coding sequence length?
    How was demographic inference done for the different models via the different programs?,
    etc.
    Can someone write this up?}
    
    % We used three software packages to infer the distribution of fitness effects (DFE) from genetic data:
    % \dadicli \citep{Huang2023,kim2017inference}, \polydfe \citep{tataru2020polydfe}, and \grapes \citep{galtier2016adaptive}.
    % Each software package was run using the same set of simulations.
    % For these analyses we used the exon annotations from the human and vaquita porpoise genomes respectively.
    % A key distinction between our analyses and those done in practice
    % is that we have complete knowledge of which sites are under selection in the simulation,
    % thus our inferences is a best case scenario. Empirical inference of the DFE is a much
    % harder problem as some mutations assumed to be free of selection
    % may in reality be under selection. As above, all methods were run using default settings.
    \igcomment{I edited the description below written by Silas. Please check to make sure it's still accurate.}
    As described above, each DFE-inference method scales selection coefficients slightly differently.
    We thus rescaled each program's output to obtain a selection coefficient ($s$) consistent with the one used by SLiM.
    The output of \grapes  was scaled by a factor of  $\mu / \theta$,
    and the output of \polydfe was saled by $2 \mu / \theta$, 
    where $\mu$ is the assumed mutation rate and 
    $\theta$ is the population-scaled mutation rate estimated by each method ($\theta = 4N_{e}\mu$).
    For \dadicli, the gamma distribution is parameterized using the shape and rate ($\alpha$ and $\nu$, respectively). 
    The mean value of the distribution, which is $s$ is transformed from $\alpha$ and $\nu$ and rescaled as $\alpha * \nu / (\theta / (4 \mu L p))$,
    where $L$ is the total number of sites available for the analysis,
    and $p$ is the proportion of sites under selection.



    \subsection*{Simulations with selective sweeps}

    To test methods for detecting selective sweeps, we simulated 100 evenly distributed 5cM
    segments of human chromosome 1.
    As recombination rate varies across the genome, the size of the region in base pairs simulated varied, however the recombination distance,
    which is critical for the linked effects of a sweep, was held constant.
    Each segment was simulated in each of the following three settings:
    (1) neutral, (2) background selection (BGS), and (3) hard sweep with BGS.
    In all three settings, we applied a genetic map from the HapMap Project
    \citep{international2007second} (\stdpopsim label \texttt{HapMapII\_GRCh38}) and
    the three-population out of Africa (OOA) demographic model from
    \cite{gutenkunst2009inferring} (\stdpopsim label \texttt{OutOfAfrica\_3G09}).
    In the two settings with BGS, we applied the DFE model with gamma-distributed selection coefficients inferred by
    \cite{kim2017inference} (\stdpopsim label \texttt{Gamma\_K17}) to exons annotated by the HAVANA group release 104
    (\cite{ensembl2018}; \stdpopsim label \texttt{ensembl\_havana\_104\_exons}).
    Hard sweeps were modeled by generating a single beneficial mutation 
    with selection coefficient of $s = 0.03$ in the center of the 5cM genomic segment
    and discarding simulations in which the beneficial mutation did not reach (or exceed) a terminal frequency of $95\%$.
    For each of the 100 genomic segments of chromosome 1, we generated 1,000 replicate datasets under each of the three settings.
    As with the whole-genome simulations (see above), we used the SLiM engine with a burn-in of $2N$ generations
    (where $N$ is the initial population size), and a scaling factor of 2.
    Each simulation generated 10 diploid genomic segments for each of the two sampled populations: YRI and CEU.
    The sequences simulated with a hard sweep and BGS were used for sweep detection,
    and sequences simulated under the other two settings were used to calibrate the thresholds of the test statistics
    used by the three sweep-detection methods (see below).

    \subsection*{Selective sweeps detection}
    %\drscomment{Maybe it makes sense to move this up to right after the sweep simulation section. Should make it easier for readers to follow.}
    %IG: I moved that section down here to bring them together
    
    Each of the three sweep-detection methods is based on a test statistic.
    Reduced nucleotide diversity is based on nucleotide diversity  ($\pi$) 
    computed directly from the underlying tree sequences using \tskit \citep{ralph2020efficiently}
    over 10 equally sized and non-overlapping subwindows.
    \sweepfinder \citep{degiorgio2016sweepfinder2} is based a composite likelihood ratio (CLR) 
    computed at 21 uniformly distributed positions along the simulated region.
    \diploshic \citep{kern2018diplos} uses supervised learning to train a sweep classifier.
    Its training phase used \texttt{discoal} \citep{kern2016discoal} to simulate 20 haploid sequences of length 1.21Mbp
    with a constant population size of $N_e=10,000$, a mutation rate of $2\times 10^{-8}$, and a recombination rate of $2\times 10^{-8}$.
    For \texttt{discoal} simulations that included selective sweeps,
    selection coefficients were drawn from a uniform distribution between 0.001 and 0.10,
    and the time of the sweep was chosen from a uniform distribution between 0 and $0.01N_e$ generations in the past.
    For soft sweeps, the frequency at which selection was introduced was drawn from a uniform distribution between $0.0001$ and $0.1$.
    We note that since the training set of \diploshic was generated under a simple model with
    constant mutation and recombination rates and constant effective population size,
    we expect it to perform suboptimally on the simulated segments of human chromosome 1,
    which were generated using more relaistic simulations (see \textbf{Simulations with selective sweeps} above).

    The three sweep-detection methods were associated with thresholds for their respective test statistics
    ($\pi$, \sweepfinder's CLR, or \diploshic's probability),
    which were determined using an empirical distribution representing a null model without sweeps.
    We considered two null models: a neutral evolution model (neutral),
    and a model with background selection from deleterious mutations in exons (BGS).
    Each null model was represented by a different set of $100,000$ simulations generated according to the appropriate model
    (see \textbf{Simulations with selective sweeps} above).
    Thus, for each test statistic we determined two thresholds corresponding to these two different null models (neutral and BGS)
    using the extreme $5\%$ quantile in each empirical distribution (and excluding regions with extremely low recombination rate).
    \igcomment{Can we say what the rec rate cutoff was and how many regions were excluded here?
    Shouldn't we have excluded these regions from the simulations altogether?}.
    Then, power (or true positive rate) to detect a sweep by a given method was determined for
    100 evenly-spaced positions on chromosome 1 as follows.
    The sweep-detection method was applied to the 1,000 
    simulated datasets generated with a hard sweep (neutral and BGS) for that position,
    and the power was set to be the proportion of simulations (out of 1,000)
    in which the test statistic exceeded the threshold determined by each of the two null models (neutral and BGS).

\section*{Data availability}\label{data_availability}


\section*{Acknowledgments}\label{acknowledgements}
We thank Tom Booker for discussions and encouragement to incorporate realistic BGS models into \stdpopsim.

\section*{Funding}
    \label{funding}
    MFR, ST, NP, and ADK were supported in part by NIH award R35GM148253.
    RNG and LNT were supported in part by NIH award R35GM149235.
    DRS was supported in part by NIH award R35GM138286.
    KEL was supported by NIH award R35GM119856.

    %other funding sources?
\printbibliography

%%%%%%%% supplementary material
\clearpage
\beginsupplement

\section*{Supplementary Material}

\begin{figure}[b!]
    \centering
    \includegraphics[width=\textwidth]{figures/PhoSin/Vaquita2Epoch_1R22/estimated_Ne_t_final.pdf}
    \caption{
    \label{fig:vaquita-demography}
    Inference of $N_e(t)$ from vaquita porpoise genomes simulated under a single population 
    model of declining population size \citep{robinson2022critically}
    with and without purifying selection on exons. 
    Inference is conducted using four methods:    
    \msmc \citep{Schiffels2020}, \stairway \citep{liu2020stairway}, \gone \citep{santiago2020recent}, and \smcpp \citep{terhorst2017robust}.
    Each plot depicts the inferred $N_e(t)$ on the three datasets simulated without selection (blue)
    and the three datasets simulated with the background effects of purifying selection on exons (orange),
    alongside the true values of $N_e$ used in simulation (black).
}
\end{figure}


% constant size DFE figure
\begin{figure}[h]
    \centering
    \includegraphics[width=\textwidth]{figures/HomSap/Constant/HomSap_Constant_Gamma_K17_ensembl_havana_104_exons_DFE_plot}
    \caption{
       Inferred versus simulated DFEs from human genomes simulated with
       a simple demographic model with one population with a constant size of $10,000$
       and a gamma-distributed DFE acting on exons (see \textbf{Methods}).
        DFE is inferred by three different methods: \grapes, \polydfe , and \dadicli.
        Mean absolute value of selection coefficient ($\lvert E(s) \rvert$, panel A) and shape parameter (panel B)
        are shown for each DFE inferred from all three simulated datase
        with median values marked by horizontal bars
        and simulated values represented by dashed horizontal lines.
    }
    \label{fig:homsap-dfe.constant}    
\end{figure}

\begin{figure}
    \centering
    \includegraphics[width=\textwidth]{figures/PhoSin/Constant/PhoSin_Constant_Gamma_R22_Phocoena_sinus.mPhoSin1.pri.110_exons_DFE_plot.pdf}
    \caption{
    \label{fig:vaquita-dfe.constant}
        Inferred versus simulated DFEs from vaquita genomes simulated with
        a simple demographic model with one population with a constant size of $3,500$
        and a gamma-distributed DFE acting on exons 
        with a relationship between the selection coefficient ($s$) and dominance coefficient ($h$) (see \textbf{Methods}).
        DFE is inferred by three different methods: \grapes, \polydfe , and \dadicli.
        Mean absolute value of selection coefficient ($\lvert E(s) \rvert$, panel A) and shape parameter (panel B)
        are shown for each DFE inferred from all three simulated datase
        with median values marked by horizontal bars
        and simulated values represented by dashed horizontal lines.
    }
\end{figure}

\begin{figure}[h]
    \centering
    \includegraphics[width=\textwidth]{figures/HomSap/Constant/HomSap_Constant_Gamma_K17_all_sites_DFE_plot}
    \caption{
    \label{fig:homsap-dfe.constant.all_sites}
    Performance of methods to infer the distribution of fitness effects (DFE) from genetic data
    under constant population size  where all sites along the simulated chromosome can experience selected and neutral mutation
    as opposed to exonic regions only.\\
    \igcomment{This fig is not cited anywhere. Do we really need it?}
    }
\end{figure}



% roc curves for sweeps
\begin{figure}
    \centering
    \includegraphics[width=0.8 \textwidth]{figures/sweeps/roc_neutral_null.pdf}
    \caption{
    \label{fig:roc-curves}
    Receiver operating characteristic (ROC) curves for the three methods for detecting sweeps.
    True positive rates (TPR) were computed across all 100,000 datasets simulated with selective sweeps,
    and false positive rates (FPR) were established by applying the methods to genomes simulated
    under neutral evolution with the same demographic history and genetic map
    used in the simulations that included sweeps (see \textbf{Methods}).
    \igcomment{Add AUCs?}
    }
    %DRS-fig: can we include AUCs here?
\end{figure}


%sweep power vs exon density
\begin{figure}
    \centering
    \includegraphics[width=0.8 \textwidth]{figures/sweeps/relationship_power_exon.pdf}
    \caption{
    %IG: copied new caption from fig 6 and adjusted.
        Power to detect selective sweeps as a function of local exon density.
        This figure shows the same power estimates shown in Figure \ref{fig:chr1_power},
        but with the 100 genomic segments sorted according to their
        exon density instead of position along chromosome 1.
        \igcomment{Specify exactly how exon density is computed (\% of bases in exons?) and how the curves were computed}
        %IG: copoed from the caption of fig 5
	Genomic segments were simulated with sweeps under a three population out-of-Africa model
        \citep{gutenkunst2009inferring} and with background selection from deleterious mutations in exons.
        Three methods for detecting sweeps were applied to simulated data:
        1) \sweepfinder \citep{degiorgio2016sweepfinder2} (top row),
        2) \diploshic \citep{kern2018diplos} (middle row),
        and 3) reduced diversity ($\pi$) (bottom row).
        Power (true positive rate) is shown for these methods for the CEU and YRI
        samples (left and right respectively).
        The thresholds of the test statistics were set to control for
        $5\%$ false positive rate under a neutral null model (blue)
        and a null model with background selection from deleterious mutations in exons (red).
    }
    \label{fig:power-exon}
\end{figure}


\begin{table}[ht]
\centering
\small

\caption{\bf{Performance of DFE methods across the simulation scenarios}. 
Bold rows show the lowest mean absolute error (MAE) for the two DFE parameters
for each species.\\
\igcomment{This table is not cited anywhere. The analysis of ``all sites" is also not mentioned anywhere. Do we want to add it?}
}
\begin{tabular}{lllrrrrrr}
\toprule
species ID & demography & annotation & \makecell{MAE \\ $E|s|$ \\ dadi-cli} & \makecell{MAE \\ $E|s|$ \\ grapes} & \makecell{MAE \\ $E|s|$ \\ polyDFE} & \makecell{MAE \\ shape \\ dadi-cli} & \makecell{MAE \\ shape \\ grapes} & \makecell{MAE \\ shape \\ polyDFE} \\
\midrule
HomSap & Constant & all sites & 0.0027 & 0.0111 & \bf{0.0018} & \bf{0.00071} & 0.029 & 0.014 \\
HomSap & Constant & exons & \bf{0.0012} & 0.0101 & 0.0023 & 0.030 & 0.0086 & \bf{0.0068} \\
HomSap & OOA 5R19 & all sites & 0.012 & 0.012 & \bf{0.0074} & \bf{0.027} & 0.055 & 0.035 \\
HomSap & OOA 5R19 & exons & 0.014 & 0.0111 & \bf{0.0072} & 0.031 & 0.051 & \bf{0.024} \\
PhoSin & Constant & all sites & \bf{0.024} & 0.025 & 0.024 & 0.21 & 0.24 & \bf{0.19} \\
PhoSin & Constant & exons & 0.024 & 0.025 & 0.024 & \bf{0.23} & 0.25 & 0.23 \\
PhoSin & Vaquita2Epoch 1R22 & all sites & \bf{0.024} & 0.025 & 0.024 & 0.20 & 0.23 & \bf{0.18} \\
PhoSin & Vaquita2Epoch 1R22 & exons & 0.024 & 0.025 & \bf{0.024} & \bf{0.18} & 0.23 & 0.21 \\
\bottomrule
\end{tabular}
\label{tab:dfe_table}
\end{table}

\stopsupplement
\end{document}
